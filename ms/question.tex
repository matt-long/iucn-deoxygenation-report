%!TEX program = pdflatex
\documentclass[draft,linenumbers]{report_chapter}

\headertext{Draft Chapter 3 for ``Ocean deoxygenation: everyone's problem
Causes, impacts, consequences and solutions''}

%-- macros
\newcommand{\note}[1]{\textcolor{red}{\footnotesize\textbf{[#1]}}}

\begin{document}
\title{Chapter 3: Oxygen Projections for the Future}

\authors{
Matthew C. Long \affil{1},
Takamitsu Ito \affil{2},
Curtis Deutsch \affil{3}
}
\affiliation{1}{Climate and Global Dynamics Laboratory, National Center for Atmospheric Research, Boulder, Colorado, USA.}
\affiliation{2}{Georgia Institute of Technology, Atlanta, Georgia, USA.}
\affiliation{3}{School of Oceanography, University of Washington, Seattle, Washington, USA.}

%-------------------------------------------------------------------------------

- What is ocean deoxygenation; how is it linked to climate warming?

- Over what timeframe do we expect ocean deoxygenation to manifest?

- Do current observations suggest that deoxygenation is evident now?
	-- Climate variability impacts ocean oxygen, complicating detection of trends.

- What is the distribution of oxygen in the present day climate?

%-----------------------------------------------------------------------------------
\section{Oxygen projections}

Figure: global oxygen projections from CESM-LE and CMIP5
	-- basic structure of decline
	-- CESM-LE is a reasonable model, so we will base in depth analysis on this solution




- What is the nature of model projections for ocean oxygen loss?
	-- Where is the ocean expected to lose oxygen; how much will it lose and how fast will it lose it?
		--- What are typical changes concentrations?
		--- Where are impacts largest?
		--- What happens to OMZs?

	-- How robust are model projections?
		--- Do the models agree?
		--- Are they based on sounds principles?
		--- What is the relative contribution of structural uncertainty and natural variability to total uncertainty in future projections?

	-- What is the long-term future for ocean oxygen under persistent green house gas emissions?

- What terms contribute to ocean deoxygenation?
	-- Does a reduction in photosynthesis matter?
	--


%-----------------------------------------------------------------------------------
\section{Comparisons with the past}
- How does the future of ocean oxygen bear resemblance to the state of the ocean over Earth's history?
	-- The Permian extinction presents an analog....

%-----------------------------------------------------------------------------------
\section{Habitat of the future}
- What are the implications of ocean deoxygenation for the character of future marine environments?
	-- What organisms are most vulnerable (vulnerability: product of exposure and tolerance)?
	-- Will deoxygenation lead to habitat fragmentation?
	-- Will oxygen minimum zones expand leading to a perturbed nitrogen cycle?
	-- Will anoxic events become more frequent?

%-----------------------------------------------------------------------------------
\section{Action}
- Can we avoid deoxygenation?

- Is ocean deoxygenation reversible on human timescales?
	-- Does the reversibility depend on when during an emission's trajectory mitigation actions are taken?

- What is the benefit of mitigation effort?
	-- Comparison of time-to-threshold behavior in RCP4.5 and RCP8.5


	%-------------------------------------------------------------------------------
	\section*{notes}
	%-------------------------------------------------------------------------------


	Loss of habitat \citep{Keeling-Kortzinger-etal-2010}.

	Detection \citep{Long-Deutsch-etal-2016}


	The extra heat in the ocean has accumulated in the near surface, while deep ocean warming proceeds more slowly.
	As ocean surface warming continues to outpace warming at depth, the vertical density gradient will continue to increase, leading to enhanced stratification.
	Stratification is likely to be further enhanced by increased freshwater inputs to the surface ocean associated with an accelerated hydrologic cycle \citep{ref}.
	Stronger stratification inhibits vertical mixing and transport between surface waters, which are in direct contact with the atmosphere, and those at depth.

	Direct versus indirect effects of warming

	Deoxygenation is primarily driven by ocean heat uptake \citep{Plattner-Joos-etal-2002,Bopp-Quere-etal-2002,Keeling-Kortzinger-etal-2010} and appears to scale approximately linearly with ocean heat uptake.

	Oxygen declines have been observed in various oceanic regions, \citep{Stramma-Johnson-etal-2008}, but it is challenging to make definitive attributions statements regarding mechanisms driving trends from current observations \citep{Long-Deutsch-etal-2016}.

	Atmospheric oxygen measurements, combined with independent estimates of ocean carbon uptake, suggest that ocean heat uptake has driven ocean oxygen loss and an excess of oxygen in the atmosphere \citep{Manning-Keeling-2006,Keeling-Manning-2014}.
	% what is the ToE of atmospheric O2?


	Oxygen declines have been observed in coastal regions \citep{Gilbert-Rabalais-etal-2010}, driven primarily by eutrophication associated with nutrient loading \citep{Rabalais-Diaz-etal-2010}.

	 Coastal hypoxia presents a forecast challenge, requiring relatively high resolution models; however, offshore boundary conditions can be prescribed following idealized scenario or large-scale model predictions \citep{Bianucci-Fennel-etal-2016}.

	Surface warming depends on emissions \citep{reference}
	Ocean is slightly delayed relative to atmosphere
	CESM in the context of CMIP5 climate sensitivity

	Distribution of warming: greater warming in the tropics and at Northern high-latitudes; delayed warming in the Southern Ocean.
	Reductions in sea ice in the north

	The remineralization depth of sinking organic matter exerts a substantial lever on atmospheric CO$_2$ concentrations \citep{Kwon-Primeau-etal-2009}.
	Reductions in export production are expected with climate warming due to nutrient limitation \citep{Steinacher-Joos-etal-2010}.

	Stratification is a function of the rate of warming \citep{Capotondi-Alexander-etal-2012}.

	Observed loss of thermocline oxygen outside of OMZs \citep{Stramma-Oschlies-etal-2012,Helm-Bindoff-etal-2011}

	Stratification and warming are likely to drive mid-latitude deoxygenation, however, the future of hypoxic and suboxic waters is less certain.

	Changes in upwelling systems, which support some of the most productive fisheries in the world are of particular concern, especially in the developing world.

	Ocean models predict declines in the global ocean oxygen inventory of 3--4\% under the RCP8.5 scenario \citep{Bopp-Resplandy-etal-2013}.


	Ocean anoxic events.

	Oxygen is required for high acuity vision in marine animals, thus deoxygenation may drive habitat compression through limiting vision \citep{McCormick-Levin-2017}.

	The most immediate and obvious impacts of climate warming on the ocean is increasing surface temperatures and sea level rise resulting from the associated thermal expansion \citep{Rhein-Rintoul-etal-2013}.


	Indeed, on land, these processes are tightly coupled; oxygen production and consumption balance nearly exactly, leaving little residual oxygen available for accumulation in the atmosphere.
	It is in the ocean, however, where a tiny fraction of organic matter is buried in the sediments,  that oxygen production and consumption are slightly imbalanced, such that residual oxygen is allowed to accumulate in the atmosphere.


	Figure: zonal mean oxygen flux in control climate?
		-- notion of a cycle at equilibrium

		%-------------------------------------------------------------------------------
		\subsection{Variability}\label{loc:historical-var}
		%-------------------------------------------------------------------------------

		\note{place holder for now...not sure how much to put here...key point is the distinction between natural and forced trends}

		Internal variability is a natural, intrinsic feature of the coupled atmosphere-ocean-land system, arising from nonlinear dynamical processes, and interactions between climate system components that integrate forcing over different timescales \citep{Hasselmann-1976}.
		The broad-band fluctuations of the climate system impart variability to oceanic oxygen distributions by inducing thermally-driven surface anomalies, modulating surface-to-depth exchange, and altering the structure of the upper ocean environment, thereby impacting organic matter production and subsequent $OUR$ in the interior \citep{Ito-Deutsch-2010,Deutsch-Brix-etal-2011}.
		A recent study using historic data and repeat sections in the Labrador Sea, for instance, linked the regional oxygen variability to deep ventilation events driven by variation in convective mixing occurring on decadal or shorter timescales \citep{van-Aken-Femke-de-Jong-etal-2011}.
		Oxygen in the subpolar North Pacific also displays significant variability \citep{Deutsch-Emerson-etal-2006}, which has been linked to fluctuations in the winter-time atmospheric forcing of the ocean \citep{Andreev-Baturina-2006}.

%-----------------------------------------------------------------------------------
\bibliography{references.bib}

\end{document}
